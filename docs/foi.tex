% Postavke dokumenta i osnovne informacije

\documentclass[a4paper,12pt]{foi}

\renewcommand{\brojAutora}{1} % Broj autora rada (max. 6)

\renewcommand{\naslov}{Praćenje aktivnosti na društvenim mrežama pomoću reaktivnih agenata} % Naslov rada

\renewcommand{\mentor}{Doc. dr. sc. Markus Schatten} % Ime i prezime mentora

\renewcommand{\autorA}{Nikola Majcen} % Ime i prezime autora
\renewcommand{\brIndeksaA}{44445-15/R} % Broj indeksa autora

\renewcommand{\vrstaRada}{Seminarski rad}
\lfoot{Kolegij: Višeagentni sustavi}

% Sadržaj

\begin{document}

\maketitle

\tableofcontents

\thispagestyle{empty}

\setcounter{page}{0}

\onehalfspacing

% Uvod

\chapter{Uvod}

Tema ovog seminarskog rada je praćenje aktivnosti na društvenim mrežama pomoću reaktivnih agenata, odnosno izrada jednostavne aplikacije gdje se pomoću agenata mogu pratiti objave na društvenim mrežama u realnom vremenu.

Seminarski rad uključuje teorijsku obradu agenata, ali praktični prikaz i implementaciju rješenja koje je vezano za praćenje aktivnosti na društvenim mrežama Twitter i Facebook. Praćenje na društvenim mrežama uključuje praćenje statusa i hashtag-ova. Rad je podjeljen na cjeline i to:

\begin{itemize}
\item{Opis zadatka}
\item{Implementacija rješenja}
\item{Demonstracija rješenja}
\end{itemize}


% Opis zadatka
\chapter{Opis zadatka}

% Implementacija rješenja

\chapter{Implementacija rješenja}

% Demonstracija rješenja

\chapter{Demonstracija rješenja}

% Zaključak

\chapter{Zaključak}

% Literatura

\addcontentsline{toc}{chapter}{Bibliografija}
\bibliography{foi.bib}
\end{document}
